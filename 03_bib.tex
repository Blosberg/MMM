\begin{thebibliography}{99}

\bibitem{Abramson_2010}
Abramson, P. R., Aldrich, J. H., Blais, A., Diamond, M., Diskin, A., Indridason, I. H., … Levine, R. (2010). Comparing Strategic Voting Under FPTP and PR. Comparative Political Studies, \textbf{43}(1), 61–90.
%https://doi.org/10.1177/0010414009341717

\bibitem{Banducci_Karp}
Banducci, S.A. and Karp, J.A. Perceptions of Fairness and Support for Proportional Representation.
\emph{Political Behavior} \textbf{21}, 217–238 (1999). https://doi.org/10.1023/A:1022083416605

\bibitem{Broadbent_poll}
David Coletto and Maciej Czop,
{\color{blue} \href{https://d3n8a8pro7vhmx.cloudfront.net/broadbent/pages/4770/attachments/original/1448994262/Canadian_Electoral_Reform_-_Report.pdf?1448994262}{Canadian Electoral Reform}},
{\emph{Public Opinion on Possible Alternatives} }, December 2015

\bibitem{Blais_1990}
Andre Blais and Kenneth Carty,
`Does Proportional Representation Foster Voter Turnout?'
European Journal of Political Research \textbf{18}(2):167-181, (1990)

\bibitem{Blais_2006}
Blais, Andr\'e and Bodet, Marc Andr\'e. . Does Proportional Representation Foster Closer Congruence Between Citizens and Policy Makers? Comparative Political Studies, \textbf{39}(10): 1243–1262 (2006)
% https://doi.org/10.1177/0010414005284374

\bibitem{Cox}
Cox, G. W. and Fiva, J. H. and Smith, D. M.  The Contraction Effect: How Proportional Representation Affects Mobilization and Turnout. The Journal of Politics, \textbf{78}(4), 1249–1263 (2016)
% doi:10.1086/686804

\bibitem{Fiva}
Fiva, J. and Hix, S. Electoral Reform and Strategic Coordination.
British Journal of Political Science, 1-10. (2020)
%doi:10.1017/S0007123419000747

\bibitem{Geys_2006}
Geys, Benny. 2006. “Explaining Voter Turnout: A Review of Aggregate-
level Research.” Electoral Studies \textbf{25}(4): 651.

\bibitem{Horowitz}
Horowitz, Donald L. "Electoral Systems: A Primer for Decision Makers."
Journal of Democracy \textbf{14}(4): 115-127 (2003).
%doi:10.1353/jod.2003.0078.

\bibitem{Karp_Banducci}
Jeffrey A. Karp and Susan A. Banducci
`The Impact of Proportional Representation on Turnout: Evidence from New Zealand.',
\emph{Australian Journal of Political Science}, \textbf{34}(3),363-377 (1999)

\bibitem{Lakeman}
E. Lakeman,
{ \emph{Power to Elect} }, Suffolk, U.K.: The Electoral Reform Society (1982) pp 59-63.

\bibitem{Leadnow_environics}
Environics,
{\color{blue} \href{http://www.votetogether.ca/pages/localpolling/}{`13 Swing Riding Polls - Wave One'}}, August 19, 2015

\bibitem{Leduc}
Leduc, L. The Failure of Electoral Reform Proposals in Canada. Political Science, \textbf{61}(2), 21–40. (2009)
%https://doi.org/10.1177/00323187090610020301

 \bibitem{Die_Zeit_negative_vote}
{\color{blue} \href{http://www.zeit.de/2005/39/Wahl\_paradox}{`Viele Farben, keine Wahl'}},
 { \emph{Die Zeit} }, September 22, 2005


%%%%%
\bibitem{Boston}
Boston, Jonathan and Church, Stephen and Bale, Tim, The Impact of Proportional Representation on Government Effectiveness: The New Zealand Experience,
Australian Journal of Public Administration,
\textbf{62}(4): 7-22 (2003)
%url = {https://onlinelibrary.wiley.com/doi/abs/10.1111/j..2003.00345.x},
%eprint = {https://onlinelibrary.wiley.com/doi/pdf/10.1111/j..2003.00345.x},
%abstract = {It is often claimed that proportional representation (PR) undermines government effectiveness, including decisional efficacy, fiscal prudence, electoral responsiveness and accountability. Drawing on New Zealand's experience since the introduction of a mixed-member proportional (MMP) electoral system in 1996, this article examines the impact of the new voting system on government effectiveness. Although government durability has been substantially reduced and the policy-making process has become more complex, governments under MMP appear to be no less able to address major policy problems or respond to changing economic circumstances. Moreover, New Zealand has maintained continuous fiscal surpluses under MMP — a radical departure from the protracted, and often large, deficits that characterised the previous two decades under a majoritarian electoral system.},
%doi:10.1111/j..2003.00345.x,
%%%%%


\bibitem{Record}
 Michael Taube
{\color{blue} \href{http://www.therecord.com/opinion-story/6692466-liberals-electoral-reform-would-stack-the-deck-in-their-favour/}{ \emph{Liberals electoral reform would stack the deck in their favour} } },
Waterloo Region Record, May 27, 2016

\bibitem{Selb}
 Selb, P. (2009). A Deeper Look at the Proportionality—Turnout Nexus. Comparative Political Studies,
 \textbf{42}(4), 527–548.
 %https://doi.org/10.1177/0010414008327427


\bibitem{Stratmann}
Thomas Strattman
{\color{blue} \href{https://www.researchgate.net/profile/Thomas_Stratmann2/publication/228978716_Party-Line_Voting_and_Committee_Assignments_in_the_German_Mixed_Member_System/links/0c96052d411ad2b0ae000000/Party-Line-Voting-and-Committee-Assignments-in-the-German-Mixed-Member-System.pdf}{Party-Line Voting and Committee Assignments in the German Mixed Member System}}
Department of Economics, George Mason University

\bibitem{Irish_howto_vote_doc}
Department of the Environment, Community and Local Government, Ireland
{\color{blue} \href{https://www.laois.ie/wp-content/uploads/Guide-to-Irelands-Electoral-System-2.pdf}{Guide to Ireland’s PR-STV Electoral System}}

\end{thebibliography}
