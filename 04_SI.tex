
\clearpage
\renewcommand{\theequation}{S.\arabic{equation}}
\renewcommand{\thesection}{S.\arabic{section}}
\setcounter{section}{0}
\setcounter{equation}{0}
% \setcounter{page}{1}
% \pagenumbering{roman}

\section*{Supplemental Material}

One challenge in creating proportionality in mixed-member (MM) systems is that there is no fair or practical way of \emph{removing} representatives from parties that are over-represented. 
Seats can be added, however, and so the additional member system (AMS) is another name for MM that highlights how MPs are added to an existing body to approach proportionality. 
A common method for doing this uses the D'Hondt highest averages method, which we briefly review now. None of the mathematical details presented here are necessary for the average voter to either (a) cast their ballot, or (b) appreciate the correspondence in proportionality between popular vote and final seat distribution.

Assume parliament must represent $N$ constituencies, after an election involving $M$ major parties. A `major' party is defined as any party whose popular support exceeds the 5\% threshold on the 2nd ballot\footnote{Minor parties are ignored, so unless otherwise stated, `major' is implied. Likewise, a vote for a `party' refers to the 2nd ballot.}. 
The index  $m \in \left\{0, .., M-1\right\} $  refers to each of the $M$ parties specifically, and quotients $Q_{m,j}$ for each party $m$ are defined as 

\begin{equation}
\label{eq:DhondtSupp}
Q_{m,j} = \frac{V_m}{j},
\end{equation}

where $j \in \left\{ 1,.., \infty \right\}$ is an index\footnote{Conventionally, this might be written as $j+1$ in the denominator, with initial value $j=0$; the above has been chosen with initial value $j=1$ for readability}, and  $V_m$ is the number of votes for party $m$, out of a total of $\hat{V}$. Let $V^I$ denote the sum of 2nd ballots that were either spoiled, or cast for minor parties that failed to meet the 5\% threshold, thus: 
\begin{equation}
\label{eq:sum_Vm}
V^I + \sum_m V_m = \hat{V}.
\end{equation}
Let $C_m$ represent the number of direct constituent seats that party $m$ was awarded from the total $N$, and $C^I$ the number of seats won by independent candidates, or candidates from minor parties,
\begin{equation}
\label{eq:sum_Cm}
C^I + \sum_m C_m = N.
\end{equation}

In practice, $C_m$ and $V_m$ will clearly be correlated, since voters tend to prefer candidates and parties along similar ideological lines. However, since $C_m$ is determined entirely by the first ballot, and $V_m$ refers exclusively to the second, these quantities are, in principle, independent.

The D'Hondt highest averages method is predicated on the assumption that for a parliament of size $\hat{N}$, the largest $\hat{N}$ quotients are awarded seats. 
For reasons described in the main text, we seek the smallest value of $\hat{N}$ for which this is possible, while still including all candidates directly elected from a constituency.
Clearly this will require $\hat{N}\ge N$, and each party will be awarded a number of supplementary seats $S_m$ (as yet undetermined) to establish their total seat count 

\begin{equation}
\label{eq:sum_Sm}
C^I + \sum_m\left( S_m +C_m\right) = \hat{N} \ge N.
\end{equation}

%-----------
The index $m$ is arbitrary, and thus we define it in order of relative constituent representation. That is to say, $m=0$ refers to the most initially over-represented party:
\begin{align}
\label{eq:most_overrep}
\frac{C_0/N}{V_0/\hat{V}} &\ge& \frac{C_m/N}{V_m/\hat{V}}, \\
\frac{C_0}{V_0} &\ge& \frac{C_m}{V_m} && \forall \,\, m \neq 0.
\end{align}

The quotient $Q_{0,C_0}$ is then the lowest quotient that corresponds to a constituent seat. To see this, note that from \ref{eq:DhondtSupp}, we have 

\begin{equation}
\label{eq:QmCm}
Q_{m,C_m} = \frac{V_m}{C_m} \ge \frac{V_0}{C_0} \,  \forall \, m,
\end{equation}
where all $C_m,V_m$ are positive integers. For proportionality, we must then allocate an additional $S_m$ seats to each party $m \neq 0$. If quotients are selected until the  threshold defined by $Q_{0,C_0}$ is met, we are left with

\begin{align}
\label{eq:QmSm}
Q_{m,C_m+S_m} = \frac{V_m}{C_m+S_m} &\to& \frac{V_0}{C_0}^+ \\ 
{C_m+S_m} &\to& \frac{V_mC_0}{V_0}^-.
\end{align}

To see how this imposes proportionality, first imagine the simplified case where $C^I=V^I=0$ (i.e., no seats won by independent candidates, and all 2nd ballot votes cast for major parties). Here, party $m$'s share of seats approaches:

\begin{align}
\label{eq:seatshare}
\frac{C_m+S_m}{ \sum\limits_k\left( C_k+S_k) \right)} &\to& \frac{V_m C_0/V_0^-}{ \sum\limits_k V_k C_0/V_0^-} \\
&\to& \frac{V_m}{\hat{V}}^-,
\end{align}
that is, party $m$'s share of the popular vote. 

This simplified case is actually not far removed from realistic elections, where independent seats are relatively rare (i.e., $C^I=0$, frequently), and recent elections have shown minor parties obtaining only a few percentage points of popular support ($V^I \approx 0$). 
Nevertheless, our proposal must be robust and precise. Accounting for independent seats and votes, however, requires subtle discussion that is not limited to mathematics, as we see in the following section. 

%====================================================

\section{Independent and Spoiled Ballots}
\label{sec:outliers}
Consider a hypothetical minimal `election' with 100 total ballots, 95 of which were valid, the other five of which were spoiled. Suppose party $X$ obtained 45 votes and remained under-represented against parties $Y$ and $Z$, who together obtained another 45 votes. The remaining 5 valid votes were cast for various small parties ($\alpha,\beta,$ etc. \ldots), each of which individually fell below the 5\% threshold required for supplementary seats.
What then is the `proportional' share of seats for party $X$?
Traditional MM would suggest 50\% ($X$'s share of the valid major-party votes), as supplemental seats are shared only among major parties. Alternatively, one might interpret `proportionality' to mean 47\% ($X$'s share of valid votes), or 45\% ($X$'s share of the voting electorate). 


In the 2011 election, for example, both the NDP and Liberal parties were under-represented; it is difficult to argue that someone who voted for the Rhinoceros party, for example, would want their share of the popular vote used to provide compensatory seats to either of these parties \--indeed, this voter explicitly voted \emph{against} these parties. 
Nor, however, did this voter explicitly support the over-represented Conservative party, either, suggesting a conundrum in which this voter's share of the popular vote cannot fairly be used for \emph{any} major party. 

This is especially true if such a minor party won seats in local ridings elsewhere (since \emph{any} additional MPs are contrary to the interests of the few constituent representatives this minor party was able to obtain.) 
A parsimonious approach would suggest that the share of valid 2nd ballot votes cast for minor parties should be agnostic with respect to major parties, and default towards the existing FPP result from the 1st ballot. 

Spoiled ballots present a different issue, but similar logic applies: some voters may wish to vote \emph{only} for their local candidate, without supporting \emph{any} existing party (e.g. if they are voting for an independent candidate). 
A voter who supports only that candidate would wish to prevent the addition of any supplementary MPs to avoid diluting their representative's power once in office; a spoiled 2nd ballot might then best serve to obstruct the addition of \emph{all} supplemental MPs.

Given these considerations, for the above hypothetical, PMM takes the approach that if party $X$ receives 45\% of the total ballots (\emph{including spoiled ballots and ballots cast for minor parties}), then it is entitled to 45\% of the parliamentary seats. 
Spoiled, and minor-party 2nd ballots should default towards the results from the first ballot (i.e., the current FPP system), and serve to inhibit supplementary MPs altogether. 

In practice, recent elections suggest that these `outlier' votes comprise about 2\% of ballots. 

Including $V_I$ and $C_I$ in the above calculations will affect the threshold defined in the above discussion, but has no effect on the \emph{ordering} of the quotient list \--only the magnitude of underrepresentation in each case. 
As such, the algorithm remains very much the same as in the standard D'Hondt process:

\begin{enumerate}
\item All FPP winners from 1st ballots are assigned seats. 
Each major party seat is associated with a quotient from their party's list, in order. The remaining quotients from all parties are then sorted in a new list.

\item Proceeding through this list in order (starting with $S_m=0 \, \forall \, m$), we may calculate the number of seats that is owed to the party associated with the quotient in question:
\begin{align}
\left(\frac{V_m}{\hat{V}}\right) \hat{N} -(C_m+S_m)\stackrel{?}{\ge} 1.
\label{eq:Qowed1seat}
\end{align} 

One might read the left side of Eq.~\ref{eq:Qowed1seat} as `the number of seats the party should expect, based on proportionality, less the number of seats the party currently has'.
If indeed the party is owed more than 1 full seat, then both the party's supplementary seat count ($S_m$), and the size of parliament ($\hat{N}$) are incremented by one, and the next quotient is considered. 

\item This proceeds until the above condition is no longer satisfied (i.e., until every party is owed a number of seats less than 1). 
\end{enumerate}

Termination of the above process will occur when quotients are very near to the threshold $Q_0,C_0$, however, since $\hat{V}$ and $\hat{N}$ include $V^I$ and $C^I$ respectively, the cutoff will be shifted slightly.
Generally, $V^I$ will drive the shift upwards, provided there are relatively few independent seats, however the potential for $C^I$ to lower this threshold demonstrates the importance of measures against decoy lists described in the main text.

The effect of spoiled and minor-party ballots is visible in figures~\ref{fig:hypo_2011} and \ref{fig:hypo_2015} of the main text where a black line shows the exact (i.e., non-integer) number of `seats' that should be associated with each party based on their popular support. 
Each of the initially under-represented parties show a PMM projection slightly below (but within 1) of this line. The initially overrepresented party is awarded no new seats, but retains a PMM projection slightly above its corresponding marker line.
As stated in the main text, the prospect of obtaining this advantage preserves the incentive of parties to win regional races.
Regional candidates may also be under pressure to put forward a serious campaign and garner 5\% of the vote to ensure their party remains eligible for 2nd ballots in the region (a possible threshold to prevent decoy lists, as discussed in the main text). 

All this serves to differentiate PMM from traditional MM models. 
In Germany, for example, national parties have pitched election campaigns asking their supporters to \emph{only} give them their second vote, since their importance eclipses the riding races at the local level. First ballot riding races are then trivialized. 
In PMM, parties still have an incentive to win the races in their local ridings.

% This is one further respect in which PMM defaults to preserving the main features of the existing system as much as possible while fixing only the problems that have been identified within it. 

%====================================================

\section{Extreme Split-Ballot Scenarios}

Data from recent elections indicate that PMM would be sufficient to establish a representative parliament with minimal expansion, however, we can imagine pathological cases. 

Suppose an under-represented party, $X$, receives zero \--or few\-- constituency seats, and yet is awarded nearly 100\% of the second-ballot votes.
This is exceedingly unlikely in practice, and would be even more implausible, given measures against decoy-lists (see conclusion, main text).
Nevertheless, for this hypothetical scenario a truly proportional legislature would require adding an \emph{infinite} number of supplemental MPs from party \textbf{$X$}.

To avoid this problem, a hard upper limit must be established \--for example, twice the number of constituencies. In our model, this quantity $\hat N \stackrel{!}{\le} N_{\textrm{max}} = 2 N$ is set as our upper-limit to ensure that constituent MPs are never in the minority, although more conservative constraints (i.e., smaller values of $N_{\textrm{max}}$) are also possible, and have been used in other MM systems. 
The general rule remains:

\begin{align}
\label{eq:Nlimits}
N &\le& \hat{N} &\le& N_{\textrm{max}} \\
FPP &\le& PMM &\le& MM
\end{align}

The inequalities above serve to highlight that PMM will entail more seats than FPP, but fewer than traditional MM, as employed in, for example, Germany. 
Both the upper and lower limits of this system (i.e., standard FPP and MM models) have been tested and shown practical in a functioning democracy, while PMM seeks an optimal combination therein.

% Finally, the 5\% threshold has been used as a filter of national support to limit supplementary seats to sincere and credible parties. 
% The fact that this has the effect of encouraging parties to seek broad-based support throughout the country may be seen as a design feature, promoting national unity, however, it will likely be met with opposition be an established regional party such as the Bloc Quebecois. 
% A palatable compromise may be to set the threshold at "5\% support nationally, \emph{or} 10\% support in at least one province...". In practice, the BQ has consistently exceeded this threshold, and the point may be moot.
